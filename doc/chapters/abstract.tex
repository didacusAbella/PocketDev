\begin{abstract}
	Internet è diventato il centro nevralgico dell'informazione moderna, e questo è un dato di fatto che non possiamo negare. Ciononostante, il web è ancora molto lontano dalla visione che Tim Berners Lee aveva: oggi il web è un ambiente incentrato unicamente sull'accessibilità dei dati da parte degli uomini ma per quanto riguarda i computer, essi gicano un ruolo nettamente inferiore infatti un computer si limita ad indicizzare le informazioni tramite parole chiave e a scambiare informazioni tra client e server. L'idea, e quindi la nascita del Web Semantico, è quella di superare questa barriera rendendo il web un posto accessibile anche per le macchine tramite un arricchimento dei dati dal punto di vista semantico. Le conseguenze di un evoluzione di questo tipo sarebbero impattanti: I nostri motori di ricerca non sarebbero più limitati ad semplici parole chiave, ma terrebbero conto di sinonimi, omonimi e soprattutto terrebbero in considerazione il contesto della ricerca. L'obiettivo del nostro progetto è quello di applicare quello che fino ad oggi è stato standardizzato del Web Semantico per creare un applicativo Web che funga da guida al percorso formativo di chiunque voglia intraprendere un futuro lavoro nel mondo della Computer Science.
\end{abstract}
 
